\documentclass[oneside]{article}

\frenchspacing

\usepackage{geometry}                		
\geometry{columnsep=80pt}   
\geometry{paperwidth=1280pt}
\geometry{paperheight=720pt}        

\usepackage[utf8]{inputenc}
% \usepackage{fontspec}
%\setmainfont{optima}
\geometry{a4paper}                   	
\geometry{landscape}

\usepackage{ulem}

\usepackage{color}
\usepackage{amssymb}
\usepackage{extarrows}

\usepackage{/Users/V/icd/sty/vmacros}
\usepackage{/Users/V/icd/sty/tokens}

\usepackage[colorlinks=true]{hyperref}
\usepackage{graphicx}				
								


\title{Testing Kandle-based strategies using an ocaml simulator}
\author{Mangrove Research}

\begin{document}
\maketitle
% \large

You need to install ocaml. 

Eg via the \href{https://opam.ocaml.org/doc/Install.html}{opam package manager}

\section{Setup price series}

To run a test we first need to get a price series. 

There are two options.

We can use a csv file of historical data (see \ttt{h} function):
\begin{verbatim}
let h 
~number_of_lines:nb_lines (* nb_lines to read after header *)
~filename:filename        (* path to csv file *)
~column:col               (* index of price column in file *)
~splitchar:c              (* split char in the csv file *)
= ...
\end{verbatim}

Or we can use a BM or GBM with parameters:
\begin{verbatim}
let gbrowmo
~initial_value:ival       (* initial price value *)
~drift:drift              (* drift of BM *) 
~volatility:vol           (* volatility of BM *)
~timestep:dt              (* integration step *)
~duration:duration        (* total duration *)
= ...
\end{verbatim}

Both methods output a price series as an array of floats.

\section{Setup Kandle}

To define a Kandle instance one needs:
\\ - a (relative) price grid 
\\ - an initial allocation of capital (in quote),
\\ - and the fraction $\al$ of the said quote to be swapped to base (to provide Kandle's asks)

The relative price grid is defined by picking any 2 of the following 3 parameters: 
\\- (half) number of points, 
\\- range multiplier, 
\\- gridstep (aka ratio)

The initial price (or entry price) 
used to center the grid is taken from the price series (hence the name `relative' price grid).

\section{Simulation}
A simulation is the interaction of a Kandle with a price series.

The output of a simulation run is a 6-uple:
\\- the return
\\- the number of up crossings and up exits (above range)
\\- the number of down crossings and down exits (below range)
\\- the final price of the price series

To set up a run of the simulation, 
we combine both sets of parameters (using here range and gridstep for the price grid).

Inputs:
\begin{verbatim}
let sim 
~rangeMultiplier:rangeMultiplier (* pmax/p0 = p0/pmin *)
~gridstep:gridstep               (* ratio of price grid *)
~quote:qB                        (* total budget in quote *)
~cashmix:alpha                   (* 0 <= alpha <= 1 *)
~start:start                     (* entry in position *)
~duration:duration               (* number of price moves *)
~price_series:price_series       (* series of price *)
\end{verbatim}

Outputs:
\begin{verbatim}
(mtmf /. mtmi -. 1.), 
!rebu, 
!cross_above, 
!rebd,  !cross_below, 
price_series.(start + duration - 1)
\end{verbatim}

Fig.~\ref{sim} shows an example of a 1000 runs, 10000 steps each. 

The syntax is
\begin{verbatim}
./a.out 10_000 10 1.25
\end{verbatim}
with arguments (in this order): number of repetitions, number of points on the grid, and ratio of the grid. 

The compiled code will generate a csv file \ttt{return\_vs\_final\_price\_scatter\_1.25\_10\_stopped.csv}. 

The csv file can be visualised using a python script called \ttt{hist.py} with similar syntax \ttt{python hist.py 10 1.25}.

\IG{300pt}
{png/return_vs_final_price_scatter.png}
{\label{sim} On the $x$-axis the final price, on the $y$-axis the return. In red (green) realisations with a negative (positive) return.}

\section{Simulation with exit}
There is a variant, called \ttt{sim\_stopped}, that will stop if the price down exits the range. So no need for specifying a duration.
Either the simulation will stop (an return the time of exit and price), and it will run all the way to the
end of the price series.

Inputs:
\begin{verbatim}
let sim_stopped 
 ~rangeMultiplier:rangeMultiplier 
 ~gridstep:gridstep 
 ~quote:qB 
 ~cashmix:alpha
 ~start:start 
 ~price_series:price_series 
\end{verbatim}
 
Outputs:
\begin{verbatim}
mtmf /. mtmi, (* growth rate, because more compositional *)
!rebu, !upper_exit,
!rebd, !lower_exit, 
!current_index,
!current_price
\end{verbatim}


\NB[1] \ttt{sim\_stopped} returns the growth rate $\ttt{mtmf /. mtmi}$ not the return $\ttt{mtmf /. mtmi} - 1$.

Fig.~\ref{simstopped} shows an example of a 1000 runs, 10000 steps each (under 2sec of execution, using the native compiler ocamlopt). 
We see the green and red walls at the ends of the range near the $\pm20\%$ mark.

\IG{250pt}
{png/return_vs_final_price_scatter_1.05_4_stopped.png}
{\label{simstopped} On the $x$-axis the final price, on the $y$-axis the return. The Kandle grid has 4 points (on each side of the entry price of $100$), with a ratio of $1.05$.}


\section{Simulation with reset}
There is also a \ttt{sim\_reset} variant where each time the price exits the range,
the strategy re-enters with the same grid translated to the new current price.

Fig.~\ref{simReset} shows an example where $r=1.02$, 1 offer on each side of the entry price, and a price GBM $\mu=0$, $\sig=0.08$.
One sees that the peyoff shape is very different, and looks
extremely similar to a buy and hold with half the cash invested (black line)
with some low amount of noise induced by the path-dependent crossings.


\IG{250pt}
{png/return_vs_final_price_scatter_1.02_0.08.png}
{\label{simReset} The regression line is close to that of a buy-and-hold strategy
with half the cash invested in the risky asset.}


\subsection{Analysis of the self-updating Kandle (sim reset)}
First, we notice that this is different from a simple MM strategy which would translate the grid each time the price moves (at a given frequency),
regardless of whether one of its offers was consumed or not. It would be interesting to simulate that as well. The difference is that in the 
latter case even in the absence of crossing the price stays always $\eps$ away from the current price.


We can `neutralise' the determinstic part of the pay-off by adding $-\ba S_t$ (eg by shorting a certain amount $\ba$ of the underlying on a perp market,
or by borrowing some of the underlying on a mending platform).
For large ratios (aka grid steps), there are $<1$ crossings on average and the constant is very nearly $\ba =\frac12$ (ie a complete buy-and-hold).
As $r$ decreases, the mean number of crossings increases and the constant $\ba$ seem to decrease.
Investigate emprically.

\IG{200pt}
{png/return_vs_final_price_scatter_1.002_0.07.png}
{adjusted return vs exit price (ratio = 1.002, vol = 0.07)}

\end{document}

