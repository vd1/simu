\documentclass[oneside]{article}

\frenchspacing

\usepackage{geometry}                		
\geometry{columnsep=80pt}   
\geometry{paperwidth=1280pt}
\geometry{paperheight=720pt}        

\usepackage[utf8]{inputenc}
% \usepackage{fontspec}
%\setmainfont{optima}
\geometry{a4paper}                   	
\geometry{landscape}

\usepackage{ulem}

\usepackage{color}
\usepackage{amssymb}
\usepackage{extarrows}

\usepackage{/Users/V/icd/sty/vmacros}
\usepackage{/Users/V/icd/sty/tokens}

\usepackage[colorlinks=true]{hyperref}
\usepackage{graphicx}				
								


\title{Testing Kandle-based strategies using an ocaml simulator}
\author{Mangrove Research}

\begin{document}
\maketitle

You need to install ocaml. 

Eg via the \href{https://opam.ocaml.org/doc/Install.html}{opam package manager}

\section{Setup}
\subsection{Setup price series}

To run a test we first need to get a price series. 

There are two options.

We can use a csv file of historical data (see \ttt{h} function):
\begin{verbatim}
let h 
~number_of_lines:nb_lines (* nb_lines to read after header *)
~filename:filename        (* path to csv file *)
~column:col               (* index of price column in file *)
~splitchar:c              (* split char in the csv file *)
= ...
\end{verbatim}

Or we can use a BM or GBM with parameters:
\begin{verbatim}
let gbrowmo
~initial_value:ival       (* initial price value *)
~drift:drift              (* drift of BM *) 
~volatility:vol           (* volatility of BM *)
~timestep:dt              (* integration step *)
~duration:duration        (* total duration *)
= ...
\end{verbatim}

Both methods output a price series as an array of floats.

\subsection{Setup Kandle}

To define a Kandle instance one needs:
\\ - a (relative) price grid 
\\ - an initial allocation of capital (in quote),
\\ - and the fraction $\al$ of the said quote to be swapped to base (to provide Kandle's asks)

The relative price grid is defined by picking any 2 of the following 3 parameters: 
\\- (half) number of points, 
\\- range multiplier, 
\\- gridstep (aka ratio)

The initial price (or entry price) 
used to center the grid is taken from the price series (hence the name `relative' price grid).

\section{Plain simulation}
A simulation is the interaction of a Kandle with a price series.

The output of a simulation run is a 6-uple:
\\- the return
\\- the number of up crossings and up exits (above range)
\\- the number of down crossings and down exits (below range)
\\- the final price of the price series

To set up a run of the simulation, 
we combine both sets of parameters (using here range and gridstep for the price grid).

Inputs:
\begin{verbatim}
let sim 
~rangeMultiplier:rangeMultiplier (* pmax/p0 = p0/pmin *)
~gridstep:gridstep               (* ratio of price grid *)
~quote:qB                        (* total budget in quote *)
~cashmix:alpha                   (* 0 <= alpha <= 1 *)
~start:start                     (* entry in position *)
~duration:duration               (* number of price moves *)
~price_series:price_series       (* series of price *)
\end{verbatim}

Outputs:
\begin{verbatim}
(mtmf /. mtmi -. 1.), 
!rebu, 
!cross_above, 
!rebd,  !cross_below, 
price_series.(start + duration - 1)
\end{verbatim}

Fig.~\ref{sim} shows an example of a 1000 runs, 10000 steps each. 

The syntax is
\begin{verbatim}
./a.out 10_000 10 1.25
\end{verbatim}
with arguments (in this order): number of repetitions, number of points on the grid, and ratio of the grid. 

The compiled code will generate a csv file \ttt{return\_vs\_final\_price\_scatter\_1.25\_10\_stopped.csv}. 

The csv file can be visualised using a python script called \ttt{hist.py} with similar syntax \ttt{python hist.py 10 1.25}.

\IG{300pt}
{alms/png/return_vs_final_price_scatter.png}
{\label{sim} On the $x$-axis the final price, on the $y$-axis the return. In red (green) realisations with a negative (positive) return.}

\section{Simulation with exit}
There is a variant, called \ttt{sim\_stopped}, that will stop if the price down exits the range. So no need for specifying a duration.
Either the simulation will stop (an return the time of exit and price), and it will run all the way to the
end of the price series.

Inputs:
\begin{verbatim}
let sim_stopped 
 ~rangeMultiplier:rangeMultiplier 
 ~gridstep:gridstep 
 ~quote:qB 
 ~cashmix:alpha
 ~start:start 
 ~price_series:price_series 
\end{verbatim}
 
Outputs:
\begin{verbatim}
mtmf /. mtmi, (* growth rate, because more compositional *)
!rebu, !upper_exit,
!rebd, !lower_exit, 
!current_index,
!current_price
\end{verbatim}


\NB[1] \ttt{sim\_stopped} returns the growth rate $\ttt{mtmf /. mtmi}$ not the return $\ttt{mtmf /. mtmi} - 1$.

Fig.~\ref{simstopped} shows an example of a 1000 runs, 10000 steps each (under 2sec of execution, using the native compiler ocamlopt). 
We see the green and red walls at the ends of the range near the $\pm20\%$ mark.

\IG{250pt}
{alms/png/return_vs_final_price_scatter_1.05_4_stopped.png}
{\label{simstopped} On the $x$-axis the final price, on the $y$-axis the return. The Kandle grid has 4 points (on each side of the entry price of $100$), with a ratio of $1.05$.}


\section{Simulation with reset}
There is also a \ttt{sim\_reset} variant where each time the price exits the range,
the strategy re-enters with the same grid translated to the new current price.

Fig.~\ref{simReset} shows an example where $r=1.02$, with 1 offer on each side of the entry price, and a price GBM $\mu=0$, $\sig=0.08$.
One sees that the payoff shape is very different, and looks
similar to a buy and hold with half the cash invested (black line)
with some low amount of noise induced by the path-dependent crossings.


\IG{250pt}
{alms/png/return_vs_final_price_scatter_1.02_0.08.png}
{\label{simReset} The regression line is close to that of a buy-and-hold strategy
with half the cash invested in the risky asset.}


\subsection{Discussion of price-tracking Kandle (sim reset)}
Price-tracking Kandle is subtly different from a vanilla MM strategy which translates its price grid each time the price moves (at a given frequency),
regardless of whether one of its offers was consumed or not. 
The difference is that in the latter case even in the absence of crossing the price stays always $\eps$ away from the current price.
% It would be interesting to simulate that as well. 

Todo: implement the vanilla MM strategy and measure the payoff difference (MM should be better).

Idea: 
One can `neutralise' the deterministic part of the pay-off by adding 
a correction of the type:
\AR{
(
    \hbox{lastPrice},
    \hbox{return}
)
&\longrightarrow&
    (\hbox{lastPrice},
    \hbox{return} - \ba (\hbox{lastPrice}/\hbox{initialPrice} - 1))
}
If $\ba=1/2$ we are subtracting the price-deterministic return of the buy-and-hold
where half of the initial cash is invested in the underlying.
The exact offset, to be determined by fitting a regression line
to a numerical experiment - will depend on $(r,\sig)$.

Concretely that can be done eg by shorting a certain amount $\ba$ of the underlying on a perp market,
or by borrowing some of the underlying on a mending platform. 

For large ratios $r>1.1$ (aka grid steps), 
there are $<1$ crossings on average and the constant is very nearly $\ba =\frac12$ (ie a complete buy-and-hold).
As $r$ decreases, the mean number of crossings increases and the constant $\ba$ seem to decrease.
To investigate empirically.

\IG{200pt}
{alms/png/return_vs_final_price_scatter_1.002_0.07.png}
{adjusted return vs exit price (ratio = 1.002, vol = 0.07)}

\QS[1] Is there for given $\sig$ a value $r\st(\sig)$ of $r$ that maximises the mean return? [similar to prior work done
for plain Kandle]

Todo: Try same experiments with multiple price points (not just 2)

Can we combine sim with reset with a stop loss? Eg, is it useful to stop resetting after $k$ moves,
or a certain minimum residual capital (or on a momentum signal)?

A small remark: concretely, when ruining on chain, price-tracking Kandle does not have access to the next price (as the simulation does)
only that $p_{n+1}>r p_n$;
to correct this one can use multiple offers to improve the upper bound,
and/or use an oracle at update time (eg consult another liquidity source).


\subsection{Analysis of sim reset}

Let $\al\in[0,1]$ be the cash ratio (amount of initial capital kept in quote).
Set $\al':=1-\al$ to simplify notations.

The total gain under a given price run is:
\EQ{
\Ga(\al) 
&=&
(\al+\al' r)^{n_u}
(\al+\al'/r)^{n_d}
}
where $n_u$ is the number of up-crossings, $n_d$
is the number of down-crossings along the price trajectories.

The crossing numbers are random and only depend on vol-and-grid, $\sig$ and $r$.

As
$\al+\al' r>1$ (barycenter of $1$ and $r>1$), and  
$\al+\al'/r<1$ (barycenter of $1$ and $1/r<1$),
it follows that $\Ga>1$ ie the strategy is profitable for a given trajectory as soon as $n_u>n_d$.

In partcular if $\al=\al'=\frac12$, we get:
\AR{
\log(\Ga(1/2)) 
&=&
n_u \log(\frac{1+r}2) + 
n_d \log(\frac{1+1/r}2)
\\&=&
N 
(
f_u \log(\frac{1+r}2) + 
f_d \log(\frac{1+1/r}2)
)
}
where in the last line we assume $N=n_u+n_d>0$ (not always true! especially for large ranges).

The above is $>0$ iff:
\AR{
    f_d/f_u
    &\leq&
    \dfrac{\log(1+r) -\log 2}{\log 2-\log(1+r\mo)} =: k(r)
}
where $k(r)\geq 1$ is increasing and converges to $\log(1+r)/\log(2) -1$ for large $r$.
This suggests that  larger $r$s are more likely to generate profits. To make sure we need
to study how the $f_d$, $f_u$ distribution depends on $r$. It seems unlikey but it could be that larger $r$ are biased 
downwards, ie $f_d/f_u$ is increasing function of $r$. [in sims up and down crossings 
look equally likely]

Todo: study the distribution of $f_u$ as a function of $r$ and $\sig$. [Also look at std]

We can see the effect of changing $\al$.

If $\al=0$ (base only), $\Ga(0) = r^{n_u-n_d}$. If we go full base,
we win iff the process crosses more up than down.

If $\al'=0$ (quote only), $\Ga(1) = 1$. Indeed if we never buy the underlying,
all that can happen is the execution of a bid, 
the output of which that strategy re-sells immmediately at same price (assuming no fees, no slippage),
and therefore the initial wealth never changes. Ie (modulo fees and slippage), setting $\al=0$ amounts to not playing.


\NB\ Euler-Maruyama for large $\sqrt{dt}$, large $\sig$ can lead to negative prices in a GBM;
so better use the integral form. 
\end{document}

app.hellosign.com/t/8e4190ab9fd7cb8acf86180f2fb42c5011d23163?
utm_campaign=multisigner_complete&
utm_source=default&
utm_channel=product_promo&
utm_medium=email&
utm_content=original