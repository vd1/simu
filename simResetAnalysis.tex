\documentclass[oneside,12pt]{article}

\frenchspacing
\usepackage{geometry}                		
\geometry{a4paper}                   	
\geometry{landscape}
% \geometry{columnsep=80pt}   
% \geometry{paperwidth=1280pt}
% \geometry{paperheight=720pt}        

\usepackage[utf8]{inputenc}
% \usepackage{fontspec}
%\setmainfont{optima}

\usepackage{ulem}

\usepackage{color}
\usepackage{amssymb}
\usepackage{extarrows}

\usepackage{/Users/V/icd/sty/vmacros}
\usepackage{/Users/V/icd/sty/tokens}

\usepackage[colorlinks=true]{hyperref}
\usepackage{graphicx}				
								

\title{Analysis of price-tracking Kandle}
\author{Mangrove Research}

\begin{document}
\maketitle
We explore two ideas in this note:
\\- extracting the deterministic impermanent loss term from the 
(last price, return) joint distribution of a price-tracking Kandle
\\- defining a closed-formula (dependent on number of crossings) for
the return of a price-tracking Kandle

% \section{Discussion of price-tracking Kandle (sim reset)}
\section{Extracting the deterministic component}
Idea: 
For price-tracking Kandle, as for all variants,
there is a strong correlation between the last price and the pay-off (or return).
In other words so-called `impermanent loss' dominates the return.

One can `neutralise' the price-deterministic part of the pay-off by adding 
a correction using the regression line with slope $b$, intercept $c$. 

Write $\hbox{priceChange}$ for the relative price change
$\hbox{lastPrice}/\hbox{initialPrice} - 1$:
\AR{
(
    \hbox{priceChange},
    \hbox{return}
)
&\longrightarrow&
    (\hbox{lastPrice},
    \hbox{return} - (b\times\hbox{priceChange} + c))
}
If $b=1/2$, $c=0$ we are subtracting the price-deterministic return of the buy-and-hold
where half of the initial cash is invested in the underlying.
The exact offsets $b$, $c$, to be determined by fitting a regression line
to a numerical experiment depend on $(r,\sig)$.

Concretely that can be done eg by shorting a certain amount $b$ of the underlying on a perp market,
or by borrowing some of the underlying on a mending platform. What does $c$ represent: 
the deterministic cost of financing the strategy (a sort of premium)?

For large ratios $r>1.1$ (aka grid steps), 
there are $<1$ crossings on average and the constant is very nearly $b=\frac12$ (ie a complete buy-and-hold).
As $r$ decreases, the mean number of crossings increases and the constant $b$, $c$ seem to decrease.
To investigate empirically.

\IG{200pt}
{alms/png/return_vs_final_price_scatter_1.002_0.07.png}
{adjusted return vs exit price (ratio = 1.002, vol = 0.07)}

\QS[1] Is there for given $\sig$ a value $r\st(\sig)$ of $r$ that maximises the mean return? [similar to prior work done
for plain Kandle]

Todo: Try same experiments with multiple price points (not just 2)

Can we combine sim with reset with a stop loss? Eg, is it useful to stop resetting after $k$ moves,
or a certain minimum residual capital (or on a momentum signal)?



\section{Analysis of sim reset}

Let $\al\in[0,1]$ be the cash ratio (amount of initial capital kept in quote).
Set $\al':=1-\al$ to simplify notations.

\PRO 
The total gain under a given price run is:
\EQ{
\Ga(\al) 
&=&
(\al+\al' r)^{n_u}
(\al+\al'/r)^{n_d}
}
where $n_u$ is the number of up-crossings, $n_d$
is the number of down-crossings along the price trajectories.
\ORP
Proof:
We start with a capital $v^0_B$ of cash (base),
of which we keep $\al v^0_B$, 
and buy $\al'v^0_B/p_0$ of base. 

Suppose an uc:
\AR{
\sMA{
    \al'v^0_B/p_0
    \\
    \al v^0_B
}
&\longrightarrow^c&
\sMA{
    0
    \\
    \al v^0_B + (rp_0)\al'v^0_B/p_0
}
}
Our new position is worth $v^0_B(\al+ \al'r)$,
so a growth rate $\ga_u=\al+ \al'r$.

Symmetrically for a dc:
\AR{
\sMA{
    \al'v^0_B/p_0
    \\
    \al v^0_B
}
&\longrightarrow^d&
\sMA{
\al'v^0_B/p_0 + \al v^0_B/(p_0/r)\\0
}
}
with a new position worth $v^0_B(\al'/r+\al )$, and we find $\ga_d=\al+ \al'/r$.
\qed

\NB\ Importantly, in the proof above, 
we assume that on a down-crossing,
the price at which one can resell what was just bought 
is the same price at which we have bought $p_0/r$. 
In reality, on a down-crossing the re-sell price may be lower:
$p_n>p_0/r\geq p_{n+1}$. 
There is a loss factor proportional to $p_{n+1}/(p_0/r)<1$ and we get 
$\hat\ga_d=\ga_dp_{n+1}/(p_0/r)\leq \ga_d<1$ instead.
Symmetrically,
one could have sold higher on an upcrossing with a ``manque à gagner'' proportional 
to $p_{n+1}/(rp_0)$. In the simulations we do mark to market at the real price $p_{n+1}$.
So we get:
\AR{
    \log\hat\Gamma
    &=&
    N_u \log(\al+\al' r) +
    N_d \log(\al+\al'/r) +
    \ell(\mbf p, r)
    \\
    \ell(\mbf p, r) &=&  \sum_{i\dar} \log(p_{i+1}/(p_0(i)/r))
}
where $p_0(i)$ is the last touched price, ie the mid-point of the strategy right before the
ith jump downwards.

This brings a series of remarks:

(1) The smaller the integration step $dt$, the smaller the loss factor.
In the limit where $\da t\to0^+$, $\log(p_{i+1}/(p_0(i)/r))\to 0^-$, which is paradoxical!
In that case $\da t$ has to be understood not just as an integration step,
but as a market reactivity time or a lag vs a driving price 
(eg a fee, maybe check the LvR paper to get a model of the lag~\cite{milionis2022automated})

(2) one could always renege on bids (for a cost $q$), to mitigate the loss, 
and just re-sell the [which corresponds to $\al=0$??]

(3a) we can also forward the bids (if we do not want to renege), or 
\\(3b) re-sell instantly on an other dex (the semi-book is closed for execution) 
[we may want to generate the volume]

Note that the crossing numbers are random and only depend on vol-and-grid, $\sig$ and $r$.

As
$\al+\al' r>1$ (barycenter of $1$ and $r>1$), and  
$\al+\al'/r<1$ (barycenter of $1$ and $1/r<1$),
it follows that $\Ga>1$ ie the strategy is profitable along a given price trajectory as soon as $n_u>n_d$.

In particular if $\al=\al'=\frac12$, we get:
\AR{
\log(\Ga(1/2)) 
&=&
n_u \log(\frac{1+r}2) + 
n_d \log(\frac{1+1/r}2)
\\&=&
N 
(
f_u \log(\frac{1+r}2) + 
f_d \log(\frac{1+1/r}2)
)
}
where in the last line we assume $N=n_u+n_d>0$ (not always true! especially for large ranges).

The above is $>0$ iff:
\AR{
    f_d/f_u
    &\leq&
    \dfrac{\log(1+r) -\log 2}{\log 2-\log(1+r\mo)} =: k(r)
}
where $k(r)\geq 1$ is increasing and converges to $\log(1+r)/\log(2) -1$ for large $r$.
This suggests that  larger $r$s are more likely to generate profits. To make sure we need
to study how the $f_d$, $f_u$ distribution depends on $r$. It seems unlikey but it could be that larger $r$ are biased 
downwards, ie $f_d/f_u$ is increasing function of $r$. [in sims up and down crossings 
look equally likely]

Todo: study the distribution of $f_u$ as a function of $r$ and $\sig$. [Also look at std]

We can see the effect of changing $\al$.

If $\al=0$ (base only), $\Ga(0) = r^{n_u-n_d}$. If we go full base,
we win iff the process crosses more up than down.

If $\al'=0$ (quote only), $\Ga(1) = 1$. Indeed if we never buy the underlying,
all that can happen is the execution of a bid, 
the output of which that strategy re-sells immmediately at same price (assuming no fees, no slippage),
and therefore the initial wealth never changes. Ie (modulo fees and slippage), setting $\al=0$ amounts to not playing.


\section{Examples}
The cost of financing a price-tracking Kandle (intercept) with $r=1.005$, $\sig=0.06$ is a negative return of 
$\sim 4.6\%$. See the comparison with the B\&H line which has an intercept of zero (Fig.~\ref{1.005-0.06r}).
The same cloud of points is represented with the regression line subtracted Fig.~\ref{1.005-0.06ar}.

It would be interesting to understand how the parameters of the regression line depend on $r$ and $\sig$.

Fig.~\ref{1.005-0.06h} gives the distribution of adjusted returns. 
From the figure it seems one can decompose the return into a random normal component
and a deterministic one      :
\EQ{\label{slope}
    % \hbox{return}
    ret(r,\sig) &=& 
    \mcl N(0,s(r,\sig)) + % random component
    (c(r,\sig)+b(r,\sig)(\Da p/p_0)) % deterministic component
}
where $\Da p/p_0$ is the relative price change at the end of the price sequence.

Todo: explore the coefficients in (Eq~\ref{slope}) above.

\IG{420pt}
{alms/png/return_vs_priceChange_scatter_1.005_0.06.png}
{\label{1.005-0.06r} non-adjusted returns vs relative price change ($r=1.005$, $\sig=0.06$); the dotted line is the buy-and-hold
return for half the capital invested in base ($\al=1/2$); the intercept of the regression line 
is the cost of the strategy per unit of capital provided.}

\IG{420pt}
{alms/png/areturn_vs_priceChange_scatter_1.005_0.06.png}
{\label{1.005-0.06ar} adjusted returns vs relative price change ($r=1.005$, $\sig=0.06$).}

\IG{420pt}
{alms/png/return_distributions_1.005_0.06.png}
{\label{1.005-0.06h} histograms of adjusted returns ($r=1.005$, $\sig=0.06$).}

\section{Random remarks}

A small remark: concretely, when running on chain, price-tracking Kandle does not have access to the next price (as the simulation does)
only that $p_{n+1}>r p_n$;
to correct this one can use multiple offers to improve the upper bound,
and/or use an oracle at update time (eg consult another liquidity source).

Euler-Maruyama for large $\sqrt{dt}$, large $\sig$ can lead to negative prices in a GBM;
so better use the integral form. 


Why is $\Ga$ an easier notation for growth of mtm:
\AR{
\Ga &=& \prod_i \ga_i
\\
R &=& \prod_i (1+r_i) - 1
\\
% operation binaire
r\star s &=& r+s+rs
}
if $r$, $s$ are small then $rs$ is negligible which explains why people talk with returns.

\bibliographystyle{plain}
\bibliography{bib}
\end{document}